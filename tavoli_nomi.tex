% !TEX TS-program = XeLaTeX
\documentclass[11pt]{extarticle}

\usepackage{fontspec}
\setmainfont{JosefinSans}[
Path = ./fonts/josefin/ ,
Extension = .ttf ,
UprightFont = *-Light ,
BoldFont = *-SemiBold ]

\usepackage[paperwidth=99mm,paperheight=105mm,margin=10mm]{geometry}
\usepackage{graphicx}
\usepackage{float}
\pagenumbering{gobble}

\begin{document}
%%%%%%%%%%%%%%%%%%%%%%%%%%%%%%
%% Pierre e Marie Courie
%%%%%%%%%%%%%%%%%%%%%%%%%%%%%%
\begin{center}
\topskip0pt
\vspace*{\fill}
{\Huge \textbf{Pierre e Marie Courie\\}}
\vspace*{\fill}
\end{center}
\newpage
\topskip0pt
\vspace*{\fill}
\begin{center}
Pierre e Marie Curie vinsero il premio Nobel per la fisica per i loro studi sulle radiazioni. La leggenda narra che si scambiarono biciclette come regalo di fidanzamento.\\
\begin{center}
\includegraphics[scale=0.1]{img/cuori_venn.eps}\\
\end{center}
\end{center}
\vspace*{\fill}
\newpage
%%%%%%%%%%%%%%%%%%%%%%%%%%%%%%
%% Kerninghan e Ritchie
%%%%%%%%%%%%%%%%%%%%%%%%%%%%%%
\begin{center}
\topskip0pt
\vspace*{\fill}
{\Huge \textbf{Kerninghan e Ritchie\\}}
\vspace*{\fill}
\end{center}
\newpage
\topskip0pt
\vspace*{\fill}
\begin{center}
Brian Kerninghan e Dennis Ritchie hanno creato il linguaggio di programmazione C, noto anche come K\&R, dalle iniziali dei suoi ideatori.\\
\begin{center}
\includegraphics[scale=0.1]{img/cuori_venn.eps}\\
\end{center}
\end{center}
\vspace*{\fill}
\newpage
%%%%%%%%%%%%%%%%%%%%%%%%%%%%%%
%% Watson e Crick
%%%%%%%%%%%%%%%%%%%%%%%%%%%%%%
\begin{center}
\topskip0pt
\vspace*{\fill}
{\Huge \textbf{Watson e Crick\\}}
\vspace*{\fill}
\end{center}
\newpage
\topskip0pt
\vspace*{\fill}
\begin{center}
I premi Nobel James Dewey Watson e Francis Crick scoprirono la struttura del DNA e il suo meccanismo di replicazione.\\
\begin{center}
\includegraphics[scale=0.1]{img/cuori_venn.eps}\\
\end{center}
\end{center}
\vspace*{\fill}
\newpage
%%%%%%%%%%%%%%%%%%%%%%%%%%%%%%
%% Alan Turing e Joan Clarke
%%%%%%%%%%%%%%%%%%%%%%%%%%%%%%
\begin{center}
\topskip0pt
\vspace*{\fill}
{\Huge \textbf{Turing e Clarke\\}}
\vspace*{\fill}
\end{center}
\newpage
\topskip0pt
\vspace*{\fill}
\begin{center}
Alan Touring e Joan Clarke erano due “spaccacodici” che aiutarono a vincere la seconda guerra mondiale grazie alla decrittazione di codici nazisti. Lui, già inventore della “macchina di Touring”, è il padre dei moderni computers. La loro vicenda ha ispirato il film “The imitation game”.\\
\begin{center}
\includegraphics[scale=0.1]{img/cuori_venn.eps}\\
\end{center}
\end{center}
\vspace*{\fill}
\newpage
%%%%%%%%%%%%%%%%%%%%%%%%%%%%%%
%% Tim Berners Lee e Robert Cailliau
%%%%%%%%%%%%%%%%%%%%%%%%%%%%%%
\begin{center}
\topskip0pt
\vspace*{\fill}
{\Huge \textbf{Berners Lee e Cailliau\\}}
\vspace*{\fill}
\end{center}
\newpage
\topskip0pt
\vspace*{\fill}
\begin{center}
Tim Berners Lee e Robert Cailliau inventarono il World Wide Web. La data di nascita del WWW è comunemente indicata nel 6 Agosto 1991, data in cui Berners Lee pubblicò il primo sito web.\\
\begin{center}
\includegraphics[scale=0.1]{img/cuori_venn.eps}\\
\end{center}
\end{center}
\begin{center}
Tim Berners Lee e Robert Cailliau are the authors of the World Wide Web. The birth date of the WWW is usually considered as the 6th of August 1991, when Berners Lee published the first web page.\\
\end{center}
\vspace*{\fill}
\newpage
%%%%%%%%%%%%%%%%%%%%%%%%%%%%%%
%% Charles Babbage e Ada Lovelace
%%%%%%%%%%%%%%%%%%%%%%%%%%%%%%
\begin{center}
\topskip0pt
\vspace*{\fill}
{\Huge \textbf{Babbage e Lovelace\\}}
\vspace*{\fill}
\end{center}
\newpage
\topskip0pt
\vspace*{\fill}
\begin{center}
Charles Babbage e Ada lovelace sono spesso definiti come proto-informatici. Babbage ha ideato i primi calcolatori programmabili con i quali la Lovelace ebbe l'occasione di lavorare. Babbage definì la Lovelace l'Incantatrice dei numeri. I due collaborarono in molteplici occasioni e si scambiavano lettere.\\
\begin{center}
\includegraphics[scale=0.1]{img/cuori_venn.eps}\\
\end{center}
\end{center}
\vspace*{\fill}
\newpage
%%%%%%%%%%%%%%%%%%%%%%%%%%%%%%
%% Don Norman e Tim Shallice
%%%%%%%%%%%%%%%%%%%%%%%%%%%%%%
\begin{center}
\topskip0pt
\vspace*{\fill}
{\Huge \textbf{Norman e Shallice\\}}
\vspace*{\fill}
\end{center}
\newpage
\topskip0pt
\vspace*{\fill}
\begin{center}
Don Norman e Tim Shallice sono i padri dell'ergonomia cognitica e teorizzarono e dimostrarono le due modalità del controllo dell'azione nell'ambito dello studio sull'attenzione.\\
\begin{center}
\includegraphics[scale=0.1]{img/cuori_venn.eps}\\
\end{center}
\end{center}
\vspace*{\fill}
\newpage
%%%%%%%%%%%%%%%%%%%%%%%%%%%%%%
%% Ferdinand de Saussure e Noam Chomsky
%%%%%%%%%%%%%%%%%%%%%%%%%%%%%%
\begin{center}
\topskip0pt
\vspace*{\fill}
{\Huge \textbf{De Saussure e Chomsky\\}}
\vspace*{\fill}
\end{center}
\newpage
\topskip0pt
\vspace*{\fill}
\begin{center}
Ferdinan de Saussure e Noam Chomsky sono due grandi linguisti che per primi teorizzarono una "scienza dei segni" (Semiologia) e la necessità di comprendere i significati profondi di una lingua, senza limitarsi a descriverla.\\
\begin{center}
\includegraphics[scale=0.1]{img/cuori_venn.eps}\\
\end{center}
\end{center}
\vspace*{\fill}
\newpage
%%%%%%%%%%%%%%%%%%%%%%%%%%%%%%
%% Le Cun e Bengio
%%%%%%%%%%%%%%%%%%%%%%%%%%%%%%
\begin{center}
\topskip0pt
\vspace*{\fill}
{\Huge \textbf{Le Cun e Bengio\\}}
\vspace*{\fill}
\end{center}
\newpage
\topskip0pt
\vspace*{\fill}
\begin{center}
Yann LeCun e Yoshua Bengio sono due ricercatori nell'ambito dell'informatica grafica. I due hanno collaborato ad un paper che descrive il riconoscimento di pattern (schemi o motivi che tendono a ripetersi) utilizzando le reti neurali ossia modelli matematici che sono in grado di apprendere a partire da esempi.\\
\begin{center}
\includegraphics[scale=0.1]{img/cuori_venn.eps}\\
\end{center}
\end{center}
\vspace*{\fill}
\newpage
%%%%%%%%%%%%%%%%%%%%%%%%%%%%%%
%% Comte e Durkheim
%%%%%%%%%%%%%%%%%%%%%%%%%%%%%%
\begin{center}
\topskip0pt
\vspace*{\fill}
{\Huge \textbf{Comte e Durkheim\\}}
\vspace*{\fill}
\end{center}
\newpage
\topskip0pt
\vspace*{\fill}
\begin{center}
Èmil Durkheim e Auguste Comte sono due importanti sociologi appartenenti e fondatori del positivismo (anche battezzato come "Fisica Sociale" da parte di Comte.\\
\begin{center}
\includegraphics[scale=0.1]{img/cuori_venn.eps}\\
\end{center}
\end{center}
\vspace*{\fill}
\newpage
%%%%%%%%%%%%%%%%%%%%%%%%%%%%%%
%% Claude Shannon e Warren Weaver
%%%%%%%%%%%%%%%%%%%%%%%%%%%%%%
\begin{center}
\topskip0pt
\vspace*{\fill}
{\Huge \textbf{Shannon e Weaver\\}}
\vspace*{\fill}
\end{center}
\newpage
\topskip0pt
\vspace*{\fill}
\begin{center}
Claud Elwood Shannon e Warren Weaver erano due matematici statunitensi che insieme teorizzarono un "modello matematico per la comunicazione" che troverebbe il metodo più efficace di codifica di un messaggio sulla base del canale attraverso il quale quel messaggio viene trasmesso.\\
\begin{center}
\includegraphics[scale=0.1]{img/cuori_venn.eps}\\
\end{center}
\end{center}
\vspace*{\fill}
\newpage
%%%%%%%%%%%%%%%%%%%%%%%%%%%%%%
%% Wirth e Van Rossum
%%%%%%%%%%%%%%%%%%%%%%%%%%%%%%
\begin{center}
\topskip0pt
\vspace*{\fill}
{\Huge \textbf{Wirth e Van Rossum\\}}
\vspace*{\fill}
\end{center}
\newpage
\topskip0pt
\vspace*{\fill}
\begin{center}
Niklaus Wirth e Guido van Rossum sono due informatici europei, vincitori entrambi di importanti premi come il Touring award e l'Advanced free software award. Hanno ideato rispettivamente i linguaggi di programmazione e scripting Pascal e Python.\\
\begin{center}
\includegraphics[scale=0.1]{img/cuori_venn.eps}\\
\end{center}
\end{center}
\vspace*{\fill}
\newpage

\end{document}