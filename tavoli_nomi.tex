% !TEX TS-program = XeLaTeX
\documentclass[11pt]{extarticle}

\usepackage{fontspec}
\setmainfont{JosefinSans}[
Path = ./fonts/josefin/ ,
Extension = .ttf ,
UprightFont = *-Light ,
BoldFont = *-SemiBold ]

\usepackage[paperwidth=99mm,paperheight=105mm,margin=10mm]{geometry}
\usepackage{graphicx}
\usepackage{float}
\pagenumbering{gobble}

\begin{document}
%%%%%%%%%%%%%%%%%%%%%%%%%%%%%%
%% Pierre e Marie Courie
%%%%%%%%%%%%%%%%%%%%%%%%%%%%%%
\begin{center}
\topskip0pt
\vspace*{\fill}
{\Huge \textbf{Pierre e Marie Courie\\}}
\vspace*{\fill}
\end{center}
\newpage
\topskip0pt
\vspace*{\fill}
\begin{center}
Pierre e Marie Curie vinsero il premio Nobel per la fisica per i loro studi sulle radiazioni. La leggenda narra che si scambiarono biciclette come regalo di fidanzamento con le quali andarono in viaggio di nozze.\\
\begin{center}
\includegraphics[scale=0.1]{img/cuori_venn.eps}\\
\end{center}
\end{center}
\vspace*{\fill}
\newpage
%%%%%%%%%%%%%%%%%%%%%%%%%%%%%%
%% Kerninghan e Ritchie
%%%%%%%%%%%%%%%%%%%%%%%%%%%%%%
\begin{center}
\topskip0pt
\vspace*{\fill}
{\Huge \textbf{Kerninghan e Ritchie\\}}
\vspace*{\fill}
\end{center}
\newpage
\topskip0pt
\vspace*{\fill}
\begin{center}
Brian Kerninghan e Dennis Ritchie sono i padri del linguaggio di programmazione C.\\Qualcuno sostiene che il nome "C" derivi da "Cat", l'animale preferito da Ritchie.\\
\begin{center}
\includegraphics[scale=0.1]{img/cuori_venn.eps}\\
\end{center}
\end{center}
\vspace*{\fill}
\newpage
%%%%%%%%%%%%%%%%%%%%%%%%%%%%%%
%% Watson e Crick
%%%%%%%%%%%%%%%%%%%%%%%%%%%%%%
\begin{center}
\topskip0pt
\vspace*{\fill}
{\Huge \textbf{Watson e Crick\\}}
\vspace*{\fill}
\end{center}
\newpage
\topskip0pt
\vspace*{\fill}
\begin{center}
I premi Nobel James Dewey Watson e Francis Crick scoprirono la struttura del DNA e il suo meccanismo di replicazione. La coppia è nota per il pessimo carattere e l'arroganza. Quando completarono la loro scoperta dissero infatti: "Abbiamo scoperto il segreto della vita!".\\
\begin{center}
\includegraphics[scale=0.1]{img/cuori_venn.eps}\\
\end{center}
\end{center}
\vspace*{\fill}
\newpage
%%%%%%%%%%%%%%%%%%%%%%%%%%%%%%
%% Alan Turing e Joan Clarke
%%%%%%%%%%%%%%%%%%%%%%%%%%%%%%
\begin{center}
\topskip0pt
\vspace*{\fill}
{\Huge \textbf{Turing e Clarke\\}}
\vspace*{\fill}
\end{center}
\newpage
\topskip0pt
\vspace*{\fill}
\begin{center}
Alan Turing e Joan Clarke ebbero un ruolo chiave nella vittoria degli alleati durante la seconda guerra mondiale. Turing chiese alla Clarke di sposarlo nel 1941, e lei accettò. Il fidanzamento venne rotto poco dopo, quando Turing decise di fare "coming-out".\\
\begin{center}
\includegraphics[scale=0.1]{img/cuori_venn.eps}\\
\end{center}
\end{center}
\begin{center}
Alan Turing and Joan Clarke had a key role for the victory of the alleis during the WWII. Turing asked Clarke to be merried in 1941 and she accepted. The engagement got soon interrupted when Turing decided to make "coming-out".\\
\end{center}
\vspace*{\fill}
\newpage
%%%%%%%%%%%%%%%%%%%%%%%%%%%%%%
%% Tim Berners Lee e Robert Cailliau
%%%%%%%%%%%%%%%%%%%%%%%%%%%%%%
\begin{center}
\topskip0pt
\vspace*{\fill}
{\Huge \textbf{Berners Lee e Cailliau\\}}
\vspace*{\fill}
\end{center}
\newpage
\topskip0pt
\vspace*{\fill}
\begin{center}
Tim Berners Lee e Robert Cailliau idearono il World Wide Web. Tim pubblicò il primo sito web il 6 Agosto 1991, Robert fu il primo "surfer" della storia del web.\\
\begin{center}
\includegraphics[scale=0.1]{img/cuori_venn.eps}\\
\end{center}
\end{center}
\begin{center}
Tim Berners Lee e Robert Cailliau are the authors of the World Wide Web. Tim published the first web page on the 6th of August 1991, Robert became the first "surfer" of the web history.\\
\end{center}
\vspace*{\fill}
\newpage
%%%%%%%%%%%%%%%%%%%%%%%%%%%%%%
%% Charles Babbage e Ada Lovelace
%%%%%%%%%%%%%%%%%%%%%%%%%%%%%%
\begin{center}
\topskip0pt
\vspace*{\fill}
{\Huge \textbf{Babbage e Lovelace\\}}
\vspace*{\fill}
\end{center}
\newpage
\topskip0pt
\vspace*{\fill}
\begin{center}
Charles Babbage e Ada Lovelace sono spesso definiti come proto-informatici. Babbage ha ideato i primi calcolatori programmabili con i quali la Lovelace ebbe l'occasione di lavorare. Babbage definì la Lovelace l'Incantatrice dei Numeri. Ebbero una lunga relazione epistolare.\\
\begin{center}
\includegraphics[scale=0.1]{img/cuori_venn.eps}\\
\end{center}
\end{center}
\vspace*{\fill}
\newpage
%%%%%%%%%%%%%%%%%%%%%%%%%%%%%%
%% Don Norman e Tim Shallice
%%%%%%%%%%%%%%%%%%%%%%%%%%%%%%
\begin{center}
\topskip0pt
\vspace*{\fill}
{\Huge \textbf{Norman e Shallice\\}}
\vspace*{\fill}
\end{center}
\newpage
\topskip0pt
\vspace*{\fill}
\begin{center}
Don Norman e Tim Shallice sono i padri dell'ergonomia cognitiva i quali teorizzarono e dimostrarono le due modalità del controllo dell'azione nell'ambito dello studio sull'attenzione. Famosa è la caffettiera di Don Norman con la quale, si dice, invitasse il collega a servirsi.\\
\begin{center}
\includegraphics[scale=0.1]{img/cuori_venn.eps}\\
\end{center}
\end{center}
\vspace*{\fill}
\newpage
%%%%%%%%%%%%%%%%%%%%%%%%%%%%%%
%% Ferdinand de Saussure e Noam Chomsky
%%%%%%%%%%%%%%%%%%%%%%%%%%%%%%
\begin{center}
\topskip0pt
\vspace*{\fill}
{\Huge \textbf{De Saussure e Chomsky\\}}
\vspace*{\fill}
\end{center}
\newpage
\topskip0pt
\vspace*{\fill}
\begin{center}
Ferdinan de Saussure e Noam Chomsky sono due grandi linguisti che per primi teorizzarono una "scienza dei segni" (Semiologia) e la necessità di comprendere i significati profondi di una lingua, senza limitarsi a descriverla.\\I due ebbero un rapporto professionale travagliato dovuto al carattere bipolare di Chomsky.\\
\begin{center}
\includegraphics[scale=0.1]{img/cuori_venn.eps}\\
\end{center}
\end{center}
\vspace*{\fill}
\newpage
%%%%%%%%%%%%%%%%%%%%%%%%%%%%%%
%% LeCun e Bengio
%%%%%%%%%%%%%%%%%%%%%%%%%%%%%%
\begin{center}
\topskip0pt
\vspace*{\fill}
{\Huge \textbf{LeCun e Bengio\\}}
\vspace*{\fill}
\end{center}
\newpage
\topskip0pt
\vspace*{\fill}
\begin{center}
Yann LeCun e Yoshua Bengio sono stati definiti "i padrini dell'intelligenza artificiale" e "la mafia canadese" (del deep learning). Famoso il profilo Twitter di LeCun per la sua ironia (@ylecun).\\
\begin{center}
\includegraphics[scale=0.1]{img/cuori_venn.eps}\\
\end{center}
\end{center}
\vspace*{\fill}
\newpage
%%%%%%%%%%%%%%%%%%%%%%%%%%%%%%
%% Antoine e Marie-Anne Lavoisier
%%%%%%%%%%%%%%%%%%%%%%%%%%%%%%
\begin{center}
\topskip0pt
\vspace*{\fill}
{\Huge \textbf{Antoine e Marie-Anne Lavoisier\\}}
\vspace*{\fill}
\end{center}
\newpage
\topskip0pt
\vspace*{\fill}
\begin{center}
Antoine e Marie-Anne Lavoisier sono considerati la coppia fondatrice della chimica moderna. Il loro amore e la loro ricerca vennero bruscamente fermati dalla Rivoluzione francese durante la quale Antoine venne accusato di tradimento e ghigliottinato.\\
\begin{center}
\includegraphics[scale=0.1]{img/cuori_venn.eps}\\
\end{center}
\end{center}
\vspace*{\fill}
\newpage
%%%%%%%%%%%%%%%%%%%%%%%%%%%%%%
%% Claude Shannon e Warren Weaver
%%%%%%%%%%%%%%%%%%%%%%%%%%%%%%
\begin{center}
\topskip0pt
\vspace*{\fill}
{\Huge \textbf{Shannon e Weaver\\}}
\vspace*{\fill}
\end{center}
\newpage
\topskip0pt
\vspace*{\fill}
\begin{center}
Claude Elwood Shannon e Warren Weaver insieme teorizzarono il "modello matematico per la comunicazione". Shannon era conosciuto per l'intrattenere i colleghi con show da giocoliere sul proprio monociclo; Weaver per la sua passione per il libro "Alice nel paese nelle meraviglie", di cui collezionò 160 versioni in 42 lingue.\\
\begin{center}
\includegraphics[scale=0.1]{img/cuori_venn.eps}\\
\end{center}
\end{center}
\vspace*{\fill}
\newpage
%%%%%%%%%%%%%%%%%%%%%%%%%%%%%%
%% Wirth e Van Rossum
%%%%%%%%%%%%%%%%%%%%%%%%%%%%%%
\begin{center}
\topskip0pt
\vspace*{\fill}
{\Huge \textbf{Wirth e Van Rossum\\}}
\vspace*{\fill}
\end{center}
\newpage
\topskip0pt
\vspace*{\fill}
\begin{center}
Niklaus Wirth e Guido van Rossum hanno ideato rispettivamente i linguaggi di programmazione e scripting Pascal e Python. Van Rossum è riconosciuto come "the Benevolent Dictator For Life" (o BDFL). \\Wirth viene ricordato per la sua citazione: "Whereas Europeans generally pronounce my name the right way ('Nick-louse Veert'), Americans invariably mangle it into 'Nickel's Worth.' This is to say that Europeans call me by name, but Americans call me by value."\\
\begin{center}
\includegraphics[scale=0.1]{img/cuori_venn.eps}\\
\end{center}
\end{center}
\vspace*{\fill}
\newpage

\end{document}